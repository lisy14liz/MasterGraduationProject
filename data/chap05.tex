% !TeX root = ../thuthesis-example.tex

\chapter{利用重要性采样对训练过程加速}\label{chapter5}
\section{引言}
这个章节主要讲了将重要性采用的方法应用于\ref{chapter4}中设计的模型的训练过程的加速。
参考相关的工作,搭建了一个,基于最新版本的深度学习框架的,
对于各类神经网络通用的训练加速的框架,介绍了框架的具体实现的方式,及其加速的效果。

\section{基于重要性采样的训练加速方法框架的设计}
本文中提到的训练加速框架是基于新版本Tensorflow 2.4的基础上完成,
应用了最新的Tensorflow的Keras接口。
新版本Tensorflow中集成了Tensorflow的灵活性,和Keras接口的高度抽象简单易用的特点。
以往的框架往往是基于旧版本的Tensorflow或Keras编写而成,
对于新版本的模型没办法调用框架来进行加速。
相对于以往的框架而言,
本系统结合了动态图和静态图的优点,模块逻辑清晰而且运行效率高。

而且框架适用于各类神经网络,对具体神经网络的结构没有限制和要求。


\subsection{系统设计与组成架构}
\subsection{模型结构设计}

\begin{figure}
  \centering
  \includegraphics[width=1.1\linewidth]{figures/重要性采样模型结构.png}
  \caption{重要性采样模型结构}
  \label{fig:importance-sampling-model}
\end{figure}

如图\ref{fig:importance-sampling-model}画出了应用重要性采样的模型构建。
整体模型架构是在原有的模型的基础上搭建完成的,在原来的模型的层后面又加入了额外的层,
这些层用到了原来模型的输出作为输入,同时引入了一些其他的数据如标签数据作为输入,
得到额外的输出。

主要的附加层是这四种,下面对它们进行分别的介绍。
\begin{enumerate}
  \item Loss layer
  \item Score layer:得到的结果用来...!!!,
  有几种可以选择的类型:
  "loss"、"gnorm"、
  "full\_gnorm"、"acc"
  \item Weight layer:会根据每个样本的分数计算得到样本的损失的权重,
  \item Metrics Layer
\end{enumerate}

\section{训练加速实验效果}

  在下面的实验中,应用前面提到的重要性采样的方法,对模型进行加速,并和原始模型的结果进行对比。
  为了消除随机性的影响,同时验证方法的鲁棒性,每组实验进行三次,会给出平均的结果,和三组分别的结果,进行比对。
  为了保证公平,同一次实验中,对比的两种训练方式会加载同样的模型初始化参数。

  \subsubsection{加速效果}
  \begin{figure}
    \centering
    \begin{subfigure}[b]{0.45\textwidth}
      \centering
      \includegraphics[width=\textwidth]{20210326_2324/FC Val loss for All.png}
      \caption{FC}
      \label{fig:FC Val loss for All}
    \end{subfigure}
    \hfill
    \begin{subfigure}[b]{0.45\textwidth}
        \centering
        \includegraphics[width=\textwidth]{20210326_2324/LSTM Val loss for All.png}
        \caption{LSTM}
        \label{fig:LSTM Val loss for All}
    \end{subfigure}
    \caption{三次对比实验训练加速平均效果:验证集的loss变化趋势图}
    \label{fig:importance-sampling-performance}
  \end{figure}

  图\ref{fig:importance-sampling-performance}中给出了,应用不同模型,
  在应用和不应用重要性采样的方法下,多次实验,验证集loss的变化平均趋势对比,
  直接这样看看不出加速效果。

  从\ref{fig:FC Val loss for All}和\ref{fig:LSTM Val loss for All}中加速前后的对比,
  我们可以看出,应用方法的初期,采样其实比较接近标准的采样方式,
  其实不会有太大的加速效果。

  \begin{table}
    \centering
    \caption{重要性采样的效果:数据随机划分下,全连接网络测试集上的MSE结果}
    \begin{tabular}{lclclcl}
      \toprule
      结果类型       & 重要性采样方法 & 对比实验 & MSE比例(<1为有效果)                                     \\
      \midrule
      三次实验平均   & 0.053 & 0.069 & 0.759 \\
      实验1    & 0.055 & 0.084 & 0.651                                \\
      实验2 & 0.054 & 0.057 & 0.948                                     \\
      实验3 & 0.049 & 0.067 & 0.733                  \\
      \bottomrule
    \end{tabular}
    \label{tab:fc random test mses}
  \end{table}

  \begin{table}
    \centering
    \caption{重要性采样的效果:数据随机划分下,长短时记忆网络测试集上的MSE结果}
    \begin{tabular}{lclclcl}
      \toprule
      结果类型       & 重要性采样方法 & 对比实验 & MSE比例(<1为有效果)                                       \\
      \midrule
      三次实验平均   & 0.096 &  0.147 & 0.652 \\
      实验1    & 0.097 &  0.142 & 0.681 \\
      实验2 & 0.109 & 0.124 & 0.882                                     \\
      实验3 & 0.082 & 0.175 & 0.467                  \\
      \bottomrule
    \end{tabular}
    \label{tab:lstm random test mses}
  \end{table}

  但是从\ref{tab:fc random test mses}和\ref{tab:lstm random test mses}中的结果来看,
  训练同样的epochs,明显应用了重要性采样方法进行训练加速的最终结果要好很多,最终的MSE的值的比例
  在加速后:加速前=0.6\~0.8左右。
  加速效果比较明显。

  \begin{figure}
    \begin{subfigure}[b]{0.45\textwidth}
        \centering
        \includegraphics[width=\textwidth]{20210326_2324/FC Val loss after epoch 100.png}
        \caption{FC (after 100 epochs)}
        \label{fig:FC Val loss after epoch 100}
    \end{subfigure}
    \hfill
    \begin{subfigure}[b]{0.45\textwidth}
        \centering
        \includegraphics[width=\textwidth]{20210326_2324/LSTM Val loss after epoch 100.png}
        \caption{LSTM (after 100 epochs)}
        \label{fig:LSTM Val loss after epoch 100}
    \end{subfigure}
      \caption{三次对比实验训练加速平均效果:验证集的loss变化趋势图(100个epoch开始)}
      \label{fig:importance-sampling-performance-after-100}
  \end{figure}
  需要注意的是,由于前100个epoch的对比不是很明显,
  \ref{fig:importance-sampling-performance-after-100}中的
  图\ref{fig:FC Val loss after epoch 100}和
  图\ref{fig:LSTM Val loss after epoch 100}
  对100个epoch以后的数据单独plot了出来。
  明显应用了重要性采样加速的验证集loss下降更快。

  \subsubsection{加速效果的稳定性}
  \begin{figure}
    \centering
    \begin{subfigure}[b]{0.45\textwidth}
        \centering
        \includegraphics[width=\textwidth]{20210326_2324/FC Val loss for All after epoch 100.png}
        \caption{FC (after 100 epochs)}
        \label{fig:FC Val loss for All after epoch 100}
    \end{subfigure}
    \hfill
    \begin{subfigure}[b]{0.45\textwidth}
        \centering
        \includegraphics[width=\textwidth]{20210326_2324/LSTM Val loss for All after epoch 100.png}
        \caption{LSTM (after 100 epochs)}
        \label{fig:LSTM Val loss for All after epoch 100}
    \end{subfigure}
      \caption{三次对比实验分别的训练加速效果:验证集的loss变化趋势图}
      \label{fig:importance-sampling-performance-All}
  \end{figure}

  本文中应用的重要性采样的方法会有比较稳定的加速效果,
  如图\ref{fig:importance-sampling-performance-All}中所示,
  重要性采样方法处理后的模型,
  相比原来原来的模型,应用不同的模型分别进行的三次实验中,loss下降的速度都要更快。
  多次实验结果比较一致,反映了加速效果会比较稳定,
  说明整个加速框架比较鲁棒,不太会受到随机因素的影响。

\section{本章小结}

