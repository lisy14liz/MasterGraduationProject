% !TeX root = ../thuthesis-example.tex

\chapter{国内外研究现状}
\section{预备知识}
\subsection{EMD介绍}
\subsection{自编码器介绍}
\subsection{重要性采样介绍}
\textbf{蒙特卡洛积分}

提到重要性采样首先要介绍蒙特卡洛积分的概念。

蒙特卡洛是一类算法的总称,是用随机抽样和统计模拟的方法,来进行数值计算。
它的基本做法是,做大量重复实验来统计频率,
根据伯努利大数定律,当样本数足够多时,频率会无限接近于概率,
所以理所当然,可以通过频率来估计概率。

对于求解积分$\int_a^b f(x)dx$,经典的方法是我们需要找出$f(x)$的原函数$F(x)$。
但是,在求积分的过程中,积分的原函数在很多情况下都不是很容易获得,那么我们就无法应用经典的求解积分的方法。

蒙特卡洛积分是蒙特卡洛算法的具体应用。
蒙特卡洛方法在估计$\int_a^b f(x)dx$积分时,将其表示为一个均匀随机变量的期望,如下,
\begin{equation}
    \theta=\int_a^b f(x)dx = (b-a)\int_a^b f(x) \frac{1}{b-a}dx=(b-a)E(f(X)), X\sim U(a,b)
\end{equation}
其中$U(a,b)$代表在[a,b]之间的均匀分布。从而可以通过如下算法来得到积分估计的结果:
\begin{enumerate}
    \item 从分布$U(a,b)$中产生i.i.d样本$x_1,x_2,x_3,...,x_n$;
    \item 计算$f(X)$期望的估计值$\overline{f(X)}=\frac{1}{n}\sum_{i=1}^{n} f(x_i)$;
    \item 得到$\hat{\theta}=(b-a)\overline{f(X)}$。
\end{enumerate}
容易得到,估计值$\hat{\theta}$的期望和方差分别为,

$E\hat{\theta}=\theta$,

$Var(\hat{\theta})=(b-a)^2 Var(\overline{f(X)})=\frac{(b-a)^2}{n}Var(f(X))$

但是基于均匀分布的估计方法,不能应用于无穷积分的估计,
而且当被积函数在积分区间上的分布不是很均匀时,抽样的效率会比较低。\\

\textbf{重要性采样}

前面的蒙特卡洛积分经典计算方法采用均匀分布作为加权函数,会有抽样效率的问题,
重要性采样就是一种利用合理的加权函数,提高抽样样本效率的计算方法。
和经典的计算方法相比,重要性采样的加权函数不再是均匀分布。
设随机变量$X$的概率密度函数是$g(x)$。记$Y=\frac{f(x)}{g(x)}$
\begin{equation}
    \theta=\int_a^b f(x)dx = \int_a^b \frac{f(x)}{g(x)}g(x)dx=EY
\end{equation}
再通过简单的蒙特卡洛积分方法估计$EY$:
\begin{equation}
    \hat{\theta}'=\hat{EY}=\frac{1}{n}\sum_{i=1}^{n}y_i=\frac{1}{n}\sum_{i=1}^{n}\frac{f(x_i)}{g(x_i)}
\end{equation}
此处$x_1,x_2,x_3,...,x_n$为从g(x)中抽取的样本。
用这个方法估计的参数的方差为
$Var(\hat{\theta}')=Var(Y)/n$。当$Y$为常数时方差为0。
所以,$f(x)$的选择目标应该是尽量接近$g(x)$。

\section{时间序列预测方法和常用模型}
处理时间序列的常用方法包括两类,
一类是传统的统计模型例如ETS和ARIMA,第二类是基于深度神经网络。

基于深度神经网络的模型预测方法近年来越来越具有竞争力。
传统的统计学模型依然具有其兼顾准确率高、模型相对简单、鲁棒性好、高效等优势。
然而,更复杂,更高维度和以及包含噪声的现实世界中的时间序列数据
无法用带有参数的解析方程来描述,因为动力学太复杂且未知,
传统浅层方法因为只包含一个小的非线性操作的数量,
没有能力准确地模拟这种复杂的数据。
从未标记数据中学习特征的优点是可以利用丰富的未标记数据,
并且可以学习比手工制作的特征更好的特征。
这两个优点都减少了对数据专业知识的需求。

近年来,在时间序列预测领域,也有关于标准框架的基础工作支持深度学习在时间序列预测问题上的发展。
出现了为时间序列预测研究设计的GluonTS开源框架\citep{DBLP:journals/corr/abs-1906-05264}。

!!!!!引用格式

\section{对时间序列模型的训练加速}
\subsection{模型训练加速方法研究现状}
\subsection{筛选有效样本(重要性采样)}
