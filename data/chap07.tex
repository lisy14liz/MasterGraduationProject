% !TeX root = ../thuthesis-example.tex

\chapter{总结与展望}\label{chapter6}


\section{本文总结}
工业装备状态时间序列的预测问题具有重大的应用价值和实际意义,
但是在技术上也具有一定的挑战性,同时普遍存在模型预测效果不够好、
模型泛化能力差且训练速度慢这两个核心的痛点。
本文应用深度学习的模型,针对预测效果和训练速度这两个问题,提出了具体的方案方法和改进措施。
本文的主要贡献可以被尝试归纳为数据分析与预处理、构建工业装备状态预测系统、搭建通用加速训练框架三个部分。

在数据分析与预处理这一部分,对研究的数据特点进行了分析,
构建了从数据导入与清洗、数据重采样、数据分段、滑动窗口处理、数据标准化到划分训练集和测试集
这一完整的数据处理流程,并且对重要的环节进行了研究,说明了操作的重要性,并分析了不同方法带来的效果。

在构建工业装备状态预测系统这部分,针对牵引系统设备温度监测,
设计了基于历史信息和基于电气信号两种不同工业装备的性能变化趋势预测模型。
这两种预测模型各有其优势,和各自适合的应用场景。
构建了针对以上两种预测模型的完整预测方案,
利用LSTM和CNN搭建时间序列预测网络,
针对信号高频分量预测效果不好的问题,本文结合了经验模态分解这一信号处理的方法,
而且提出不同频率的信号应用不同的采样间隔的方法,
提升预测的效果和速度。尝试应用自编码器进行相关的特征处理。

在搭建通用加速训练框架这部分中,
本文应用了重要性采样的方法对模型训练进行加速,设计和搭建了通用的加速训练的框架,
并进行实验验证了其有效性和鲁棒性。我们在实际的工业装备状态的预测网络上进行了测试,
多次实验,均能够有效缩短训练过程50\%以上。

\section{未来展望}

本文的研究还有进一步提升的空间。

在模型选择方面,本文研究和应用的模型都是从经典的模型出发,并进行组合和改进,
但是在时间序列预测这一领域上,不同模型结构的构建和应用也有了长足的发展,
可以进一步尝试例如TCN等比较新的网络结构,验证其效果,尝试进一步提升模型预测效果。

在训练加速方面,除了重要性采样,还有很多其他角度的方法可以改进模型泛化能力差且训练时间长的问题。
在本文中集中对重要性采样的方法进行了研究,但是如果要实际应用,结合多种方法的效果会更加地可观。
