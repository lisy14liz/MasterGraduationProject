% !TeX root = ../thuthesis-example.tex

\chapter{引言}\label{chapter1}

\section{研究背景与意义}
据估计,在2020年,我们每个人平均每秒会产生1.7MB的数据,
仅仅在过去的两年,就产生了世界上90\%的数据,
我们每天正在产生2.5亿字节的数据\cite{siegel2013predictive}。
在快速发展的技术的推动下,我们身边产生的数据不断地增加。
随着工业4.0时代的到来,由于现代生活各个方面的数字化,
例如,物联网(Internet of Things, IoT)技术正在不断深化发展,
时间序列数据的数据量急剧增加。

大量数据的增加,“大数据”时代的到来,带来了很多新的机遇,
也给研究人员带来了新的挑战。
我们现在在处理、存储和分析大数据的能力方面也取得了很多进展,
例如在计算集群上用分布式的方式处理和存储数据方面,及用人工智能
(AI, Artificial intelligence)和机器学习
(ML, Machine learning)的方式处理大量非结构化
数据内容的能力等,这些方面都有了很大的发展等。

现代装备的复杂性和高度集成性也导致了故障表现形式的复杂化,
而且由于设备装态受环境和工况影响比较大,
传统的基于机理的分析方式没有办法涵盖所有的故障问题。
工业大数据是智能制造和制造服务化的重要组成部分,
在制造、电力、航空、轨道交通、船舶、石油等各方面都有很多具体的应用。
工业装备状态预测,尤其是复杂装备的状态预测,
是装备状态监测及运维服务系统的基础。
基于数据的自动化的解决方案,
能够大大地降低装备的运行成本,保障装备运行的稳定性和安全性,
对制造业的跨越式的发展有着极其重要的意义。

时间序列是按照时间顺序对特定过程进行观测而形成的数值序列。
时间序列分析的目的在于通过挖掘反映数据变化规律的模式,
在理解时间序列的基础上实现分类和预测,从而支持针对相应
自然和社会现象的决策。
高效能时间序列分析在金融、生物信息、自然灾害预测、
过程控制等方面具有极其重要的作用。

针对历史IoT数据进行预测具有十分重要的意义。
一方面可以通过数据预测未来趋势,从而进行优化决策,例如针对风力发电的风力预测;
另一方面,时间序列预测也可以用于实时故障检测,
由于工业数据往往存在故障样本稀少的问题,
可以通过建立基于正常数据的预测模型,
然后通过分析预测模型(反映正常运行模式)和实际数据的残差进行故障预测。
通过数据驱动的方式重建机理,将预测式分析手段应用于装备部件、系统甚至是整车
级别的故障预测、推理和剩余寿命估计,能全面地提高设备的健康管理水平,
大大地提高运维效率,和装备的出勤率,节约运维和设备运行的成本和效率。

\section{问题描述与主要挑战}
  \subsection{问题描述及研究目标}
  \paragraph{问题描述}~{}

    本课题针对工业装备状态这一类时间序列的预测问题,研究如何训练高准确度的模型,及训练过程的加速问题。
  \paragraph{研究目标}~{}\label{goal}

    时间序列的预测问题的典型解决方法是应用传统的概率统计模型,
    由于其准确率高、模型相对简单、鲁棒性好等优点,应用广泛。
    但是传统方法特征提取自动化程度低、验证依靠经验和人工分析。
    面对海量数据,深度学习具有自动化特征提取、对机理和人工依赖程度较低的优点,
    成为未来的发展趋势。
    因此,本文采用基于深度学习的时间序列预测框架。

    然而,现有的深度学习方法的有以下三个问题:
    \begin{enumerate}
      \item 难以跟踪快速跳变数值。深度神经网络在回归问题上往往难以适应快速变化的数据,导致平均精度高、但是极值点附近误差较大的问题,然而极值点附近又往往是预测的重点。
      \item 模型训练速度慢。在用深度学习处理时间序列问题中,模型优化的计算量大,需要的样本数量多,训练速度慢。导致在很多场景下,深度学习在时间序列上的应用相较于传统方法,并没有明显的优势。
      \item 模型定制要求高。和深度学习应用到图像分类问题中不同,
      时间序列问题中数据之间的分布并没有显著的相似性。
      如果训练一个全局的深度学习模型,在具体的某一时间序列数据上,
      模型的效果可能会特别差\cite{lebedev2018speeding}。时间序列问题的模型迁移能力差,
      无法将神经网络应用于各类不同的任务。工业数据情况复杂多变,
      往往需要针对问题定制模型,甚至每台机器都需要单独的模型。
    \end{enumerate}

    深度神经网络的训练时间很长,加之需要多个不同的模型。
    训练速度极大地限制了深度学习在时间序列问题上的发展。
    针对以上提出的现有的深度学习方法存在的问题,
    本课题提出了两个的研究目标:
    \begin{enumerate}
      \item 训练出准确度能满足实际应用需求的时间序列预测模型。
      \item 快速而且有效地训练时间序列模型。
    \end{enumerate}


  \subsection{时间序列预测的挑战性}
    自然和社会现象的错综复杂决定了时间序列的复杂性,
    本课题针对的工业装备IoT数据更具有以下鲜明特点:
    \begin{enumerate}[(1)]
      \item 动态性:工业数据往往具有时变和非稳态的特点,
      即数据背后的统计分布和规律随时间变化,
      同时各种突发事件、偶然因素的影响也会造成非趋势性和非周期性的不规则变动。
      \item 多样性:需要多种模型。
      时间序列问题的数据之间的分布并没有显著的相似性。
      如果训练一个全局的模型,在具体的某一时间序列数据上,
      模型的效果可能会特别差。时间序列问题的模型迁移能力差,
      无法将神经网络应用于各类不同的任务。
      工业数据情况复杂多变,往往需要针对问题定制模型,
      甚至每台机器都需要单独的模型。
      \item 小样本:工业装备IoT数据的典型特点是故障类型呈长尾分布,
      正常运行样本极为丰富,特点故障样本稀少。
      样本的严重不均衡导致故障和非故障的分类十分困难,
      传统的监督式分类方法在这种情况下并不适用。
      \item 高维度:一方面,随着各方面硬件技术的不断发展,
      实际应用中数据的采样频率不断提高,因此时间序列的长度也不断变大,
      仅仅把时间序列看作单纯的一维向量数据来处理不可避免地会带来维数灾难等问题;
      另一方面,很多实际应用中的时间序列数据往往包含多个变量(multivariate),
      这些变量之间往往存在复杂依赖关系。
    \end{enumerate}

\section{研究内容与主要贡献}
以电网数据和告诉机车数据为研究对象,针对时间序列回归预测问题,
研究高准确度预测模型以及加速训练技术和方法。
针对章节\ref{goal}中提出的两个研究目标,研究内容分为以下几个部分。
\begin{enumerate}[(1)]
  \item 针对时间序列预测问题,调研和应用了深度神经网络模型结构和方法。
  \item 设计合理的模型训练框架,对时间序列模型的训练进行了加速。
  \item 对具体的电网问题和轨道车辆走行部问题,对数据进行了分析和处理,搭建了时间序列模型训练和预测的系统。
\end{enumerate}

针对具体的研究内容,设计的研究方案包括如下主要内容:

\begin{enumerate}
  \item 设计了基于历史信息和基于电气信号两种不同的工业装备预测方案。
  \item 应用LSTM和CNN搭建时间序列预测网络,并在此基础上对预测系统进行改进。
  利用经验模态分解进行信号分解,对不同频率信号进行分别预测;并且进一步对高低频数据采用不同的采样频率;
  尝试应用自编码器进行相关特征处理。
  \item 应用重要性采样的方法对模型训练进行加速。
  \item 设计和搭建了通用的加速框架,并实验其效果,验证其稳定性。
\end{enumerate}

\section{组织结构}
本文总共分为六个章节。

第 \ref{chapter1} 个章节引言中,对问题本身进行了阐述,包括研究背景与意义,和问题描述与挑战分析。

第 \ref{chapter2} 个章节国内外研究现状,主要是对预备知识的阐述和研究现状的调研与分析。

第 \ref{chapter3} 、 \ref{chapter4} 、 \ref{chapter5} 是文章的主要工作内容的介绍。
这三个章节中,第 \ref{chapter4} 、 \ref{chapter5} 章介绍了本文的两大块的主要内容,
分别是工业装备的预测,和利用重要性采样对训练过程进行加速。
在第 \ref{chapter3} 中则对这两部分内容中实验的设定和数据部分进行了说明,
数据部分包含数据说明、数据特点分析和数据预处理几大部分。

在最后第  \ref{chapter6} 章中,对全文的研究做了总结,并提出了未来的展望。