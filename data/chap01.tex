% !TeX root = ../thuthesis-example.tex

\chapter{引言}

\section{研究背景与意义}
时间序列是按照时间顺序对特定过程进行观测而形成的数值序列。
时间序列分析的目的在于通过挖掘反映数据变化规律的模式,
在理解时间序列的基础上实现分类和预测,从而支持针对相应自然和社会现象的决策。
高效能时间序列分析在金融、生物信息、自然灾害预测、过程控制等方面具有极其重要的作用。
近来,随着工业4.0时代的到来,
物联网(Internet of Things, IoT)技术正在不断深化发展,
时间序列数据的数据量急剧增加。

针对历史IoT数据进行预测具有十分重要的意义。
一方面可以通过数据预测未来趋势,从而进行优化决策,例如针对风力发电的风力预测;
另一方面,时间序列预测也可以用于实时故障检测,
由于工业数据往往存在故障样本稀少的问题,
可以通过建立基于正常数据的预测模型,
然后通过分析预测模型(反映正常运行模式)和实际数据的残差进行故障预测。

本课题基于课题组承担的科技部国家重点研发计划-产品服务生命周期集成平台研发(2019–2021)展开,项目编号为2018YFB1702602。

\section{问题描述与主要挑战}
\subsection{问题描述}
本课题针对时间序列预测问题,研究如何训练高准确度的模型,及训练过程的加速问题。

\subsection{时间序列预测的挑战性}
自然和社会现象的错综复杂决定了时间序列的复杂性,
本课题针对的IoT数据更具有以下鲜明特点:
\begin{enumerate}[(1)]
  \item 动态性:工业数据往往具有时变和非稳态的特点,即数据背后的统计分布和规律随时间变化,同时各种突发事件、偶然因素的影响也会造成非趋势性和非周期性的不规则变动。
  \item 多样性:需要多种模型。时间序列问题的数据之间的分布并没有显著的相似性。
  如果训练一个全局的模型,在具体的某一时间序列数据上,模型的效果可能会特别差。时间序列问题的模型迁移能力差,无法将神经网络应用于各类不同的任务。工业数据情况复杂多变,往往需要针对问题定制模型,甚至每台机器都需要单独的模型。
  \item 小样本:工业装备IoT数据的典型特点是故障类型呈长尾分布,正常运行样本极为丰富,特点故障样本稀少。样本的严重不均衡导致故障和非故障的分类十分困难,传统的监督式分类方法在这种情况下并不适用。
  \item 高维度:一方面,随着各方面硬件技术的不断发展,实际应用中数据的采样频率不断提高,因此时间序列的长度也不断变大,仅仅把时间序列看作单纯的一维向量数据来处理不可避免地会带来维数灾难等问题;另一方面,很多实际应用中的时间序列数据往往包含多个变量(multivariate),这些变量之间往往存在复杂依赖关系。
\end{enumerate}


\section{研究内容与主要贡献}
以高速机车数据和其它设备数据为研究对象,针对时间序列回归预测问题,研究高准确度预测模型以及加速训练技术和方法。研究内容分为以下几个部分。
\begin{enumerate}[(1)]
  \item 针对时间序列预测问题,调研和应用了深度神经网络模型结构和方法。
  \item 设计合理的模型训练框架,对时间序列模型的训练进行了加速。
  \item 对具体的轨道车辆走行部问题,搭建了时间序列模型训练和预测的系统。
\end{enumerate}
\section{组织结构}