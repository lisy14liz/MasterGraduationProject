% !TeX root = ../thuthesis-example.tex

% 中英文摘要和关键字

\begin{abstract}
  % 论文的摘要是对论文研究内容和成果的高度概括。
  % 摘要应对论文所研究的问题及其研究目的进行描述,对研究方法和过程进行简单介绍,对研究成果和所得结论进行概括。
  % 摘要应具有独立性和自明性,其内容应包含与论文全文同等量的主要信息。
  % 使读者即使不阅读全文,通过摘要就能了解论文的总体内容和主要成果。

  % 关键词用“英文逗号”分隔,输出时会自动处理为正确的分隔符

  利用蓬勃发展的人工智能和大数据技术,
  针对工业装备特别是复杂装备实现预测式维护是工业互联网+的重要组成部分,
  能够大幅度降低装备的运行成本、全面保障装备安全稳定运行,
  对于我国制造业转型升级、跨越式发展具有极其重要的意义。
  由于工业装备数据以传感器采集的时间序列数据为主,
  因此以上问题很大程度上体现为针对时间序列的预测问题。
  
  工业装备数据来源复杂、机理多样、场景众多,
  而且数据分布上一般具有正常运行样本量巨大单故障样本稀少的特点。
  如何根据正常运行样本进行训练并同时具备故障检测能力,
  已经成为亟待解决的问题。
  
  同时, 由于其数据特点,工业人工智能摸往往存在着模型数量大、
  训练时间长、模型泛化能力差的问题,
  如何加快模型训练速度、提供模型构造效率,也成为一个迫切需要解决的问题。 

针对以上问题,本文开展两个方面的研究工作:1)利用正常数据训练出满足实际应用需求、并能发现故障的时间序列预测模型;2)提高时间序列模型训练效率。
本文的主要贡献包括如下内容:

\begin{enumerate}
  \item 针对牵引系统设备温度监测,
  设计了基于历史信息和基于电气信号两种不同工业装备的性能变化趋势预测模型; 
  \item 构造了基于以上模型的完整预测方案,首先应用LSTM 和 CNN 搭建时间序列预测网络,
  利用经验模态分解进行信号分解,
  对不同频率信号进行分别预测;
  并且进一步对高低频数据采用不同的采样频率;尝试应用自编码器进行相关特征处理。 
  \item 研究利用重要性采样的方法对模型训练进行加速, 设计和搭建了通用加速训练框架,并且进行了实验验证,能够有效缩短训练过程50\%。
\end{enumerate}

  \thusetup{
    keywords = {时间序列预测, 深度神经网络, 重要性采样, 经验模态分解},
  }
\end{abstract}

\begin{abstract*}
  An abstract of a dissertation is a summary and extraction of research work and contributions.
  Included in an abstract should be description of research topic and research objective, brief introduction to methodology and research process, and summarization of conclusion and contributions of the research.
  An abstract should be characterized by independence and clarity and carry identical information with the dissertation.
  It should be such that the general idea and major contributions of the dissertation are conveyed without reading the dissertation.

  An abstract should be concise and to the point.
  It is a misunderstanding to make an abstract an outline of the dissertation and words “the first chapter”, “the second chapter” and the like should be avoided in the abstract.

  Keywords are terms used in a dissertation for indexing, reflecting core information of the dissertation.
  An abstract may contain a maximum of 5 keywords, with semi-colons used in between to separate one another.

  % Use comma as seperator when inputting
  \thusetup{
    keywords* = {keyword 1, keyword 2, keyword 3, keyword 4, keyword 5},
  }
\end{abstract*}

