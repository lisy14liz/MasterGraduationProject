% !TeX root = ../thuthesis-example.tex

% 中英文摘要和关键字

\begin{abstract}
  % 论文的摘要是对论文研究内容和成果的高度概括。
  % 摘要应对论文所研究的问题及其研究目的进行描述,对研究方法和过程进行简单介绍,对研究成果和所得结论进行概括。
  % 摘要应具有独立性和自明性,其内容应包含与论文全文同等量的主要信息。
  % 使读者即使不阅读全文,通过摘要就能了解论文的总体内容和主要成果。

  % 关键词用“英文逗号”分隔,输出时会自动处理为正确的分隔符

  工业装备状态预测,尤其是复杂装备的状态预测,
  是装备状态监测及运维服务系统的基础。
  基于数据的自动化的解决方案,
  能够大大地降低装备的运行成本,保障装备运行的稳定性和安全性,
  对制造业的跨越式的发展有着极其重要的意义。工业装备数据大多都是时间序列数据,
  针对时间序列的预测具有十分重要的意义。

  但是工业装备状态预测由于其数据特点,
  往往存在着训练时间长、模型泛化能力差的问题。
  所以如何快速训练模型,成为一个需要解决的问题。

  所以针对这些问题,本文提出了两个研究目标:
  \begin{enumerate}
    \item 训练出准确度能满足实际应用需求的时间序列预测模型。
    \item 快速而且有效地训练时间序列模型。
  \end{enumerate}

  本文的主要贡献包括如下内容:
  \begin{enumerate}
    \item 设计了基于历史信息和基于电气信号两种不同的工业装备预测方案。
    \item 应用LSTM和CNN搭建时间序列预测网络,并在此基础上对预测系统进行改进。
    利用经验模态分解进行信号分解,对不同频率信号进行分别预测;并且进一步对高低频数据采用不同的采样频率;
    尝试应用自编码器进行相关特征处理。
    \item 应用重要性采样的方法对模型训练进行加速。
    \item 设计和搭建了通用的加速框架,并实验其效果,验证其稳定性。
  \end{enumerate}

  \thusetup{
    keywords = {时间序列预测, 深度神经网络, 重要性采样, 经验模态分解},
  }
\end{abstract}

\begin{abstract*}
  An abstract of a dissertation is a summary and extraction of research work and contributions.
  Included in an abstract should be description of research topic and research objective, brief introduction to methodology and research process, and summarization of conclusion and contributions of the research.
  An abstract should be characterized by independence and clarity and carry identical information with the dissertation.
  It should be such that the general idea and major contributions of the dissertation are conveyed without reading the dissertation.

  An abstract should be concise and to the point.
  It is a misunderstanding to make an abstract an outline of the dissertation and words “the first chapter”, “the second chapter” and the like should be avoided in the abstract.

  Keywords are terms used in a dissertation for indexing, reflecting core information of the dissertation.
  An abstract may contain a maximum of 5 keywords, with semi-colons used in between to separate one another.

  % Use comma as seperator when inputting
  \thusetup{
    keywords* = {keyword 1, keyword 2, keyword 3, keyword 4, keyword 5},
  }
\end{abstract*}
